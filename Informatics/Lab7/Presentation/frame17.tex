\begin{frame}[t]{Анализ свойств меры Хартли}
\noindent Экспериментатор одновременно подбрасывает монету (М) и кидает игральную кость (К).
Какое количество информации содержится в эксперименте (Э)?\\
\vspace{1.5em}
\noindent\textbf{Аддитивность:}\\
\color{black}
$i(\mbox{Э})=i(M)+i(K)=>i(\mbox{12 исходов})=i(\mbox{2 исхода})+i(\mbox{6 исходов}):\ \log_x12=\log_x2+\log_x6$\\
\noindent\textbf{Неотрицательность:}\\
\color{black}
Функция $log_xN$ неотрицательно при любом $x>1$ и $N\geq1$\\
\noindent\textbf{Монотонность:}\\
\color{black}
С увеличением $p(M)$ или $p(K)$ функция $i(\mbox{Э})$ монотонно возрастает.\\
\noindent\textbf{Принцип неопределённости:}\\
\color{black}
При наличии всегда только одного исхода (монета и кость с магнитом) количество\\
информации равно нулю: $\log_x1+log_x1=0$
\end{frame}