\begin{frame}[t]{Нестрогий вывод формулы Шеннона}
	\noindent\textbf{Задача.}
	\color{black}
	Монета имеет смещённый центр тяжести. Вероятность выпадения «орла» –-- 0,25, вероятность выпадения «решки» –-- 0,75. Какое количество информации содержится в одном подбрасывании?\\
	\vspace{0.5cm}
	\noindent\textbf{Решение}
	\color{black}
	\begin{itemize}
		\item[\textbullet] Пусть монета была подброшена N раз $(N\rightarrow\infty)$, из которых «решка» выпала M раз, «орёл» — $K$ раз (очевидно, что $N=M+K$).\\
		\item[\textbullet] Количество информации в $N$ подбрасываниях: $i_N = M*i(\mbox{«решка»})+K*i(\mbox{«орёл»}).$\\
		\item[\textbullet] Тогда среднее количество информации в одном подбрасывании:\\
		$i_1=i_N/N=(M/N)*i(\mbox{«решка»})+(K/N)*i(\mbox{«орёл»})=$\\
		$=p(\mbox{«решка»})*i(\mbox{«решка»})+p(\mbox{«орёл»})*i(\mbox{«орёл»}).$
		\item[\textbullet] Подставив формулу Шеннона для i, окончательно получим:\\
		$i_1=-p(\mbox{«решка»})*\log_xp(\mbox{«решка»})-p(\mbox{«орёл»})*\log_xp(\mbox{«орёл»})\approx0,8$ бит
	\end{itemize}
\end{frame}